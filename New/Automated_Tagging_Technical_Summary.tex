
\documentclass[12pt]{article}
\usepackage{geometry}
\usepackage{amsmath}
\usepackage{hyperref}
\usepackage{listings}
\geometry{a4paper}

\title{Automated Tagging of Incident Learning Narratives: A Technical Summary}
\author{}
\date{}

\begin{document}
\maketitle

\section{Abstract}
This project aims to develop an automated system to tag and categorize incident learning entries, which are narratives submitted by users. These tags serve to identify recurring themes, issues, or critical elements within the narratives to facilitate future reviews and analyses. The system employs Natural Language Processing (NLP) techniques, specifically Term Frequency-Inverse Document Frequency (TF-IDF) and n-gram extraction.

\section{Objective}
The primary objective is to automatically parse through a database of incident learning narratives and assign tags to these entries. These tags serve as succinct summaries and categorizations of the narratives, facilitating easier future analyses, searches, and reviews.

\section{Data Source}
The data consists of an Excel file that comprises several columns, one of which is labeled '105.Narrative.' This column contains textual summaries or narratives describing various incidents and serves as the primary data source for this project.

\section{Methodology}

\subsection{Data Preprocessing}
The initial step in the data pipeline involves cleaning and preparing the text data for feature extraction. The textual narratives are converted to lowercase, and tokens that are not alphabetic are filtered out to simplify the subsequent analysis.

\begin{lstlisting}[language=Python]
df['Preprocessed_Narratives_FullWords'] = df['105.Narrative'].apply(preprocess_text)
\end{lstlisting}

\subsection{Feature Extraction}

\subsubsection{Single-Word Tags}
The system identifies the top 50 most frequently occurring words across all narratives. These high-frequency words are considered as potential single-word tags.

\begin{lstlisting}[language=Python]
filtered_top_50_words_for_tags = [word for word, freq in words_freq[:50]]
\end{lstlisting}

\subsubsection{Context-Based Multi-Word Tags}
The system also allows for predefined, contextually relevant multi-word tags. Examples include but are not limited to 'Dosimetric Errors,' 'Patient Delayed,' and 'Treatment Planning.'

\begin{lstlisting}[language=Python]
context_based_multi_word_tags = [...]
\end{lstlisting}

\subsubsection{N-grams}
In addition to single-word and multi-word tags, the system also extracts frequent n-grams (bi-grams, tri-grams, and four-grams). An n-gram is a contiguous sequence of n items from a given sample of text or speech. For this project, n-grams that occur at least five times across all narratives are considered.

\begin{lstlisting}[language=Python]
frequent_bigrams = extract_frequent_ngrams(df['Preprocessed_Narratives_FullWords'], 2)
\end{lstlisting}

\subsection{Tagging Mechanism}

\subsubsection{TF-IDF Vectorization}
The Term Frequency-Inverse Document Frequency (TF-IDF) technique is employed to convert the narratives into a mathematical form. This vectorization technique produces a high-dimensional vector for each narrative, where each dimension corresponds to one potential tag. The TF-IDF score of each tag within each narrative is calculated to gauge its relevance.

\begin{lstlisting}[language=Python]
vectorizer = TfidfVectorizer(vocabulary=all_possible_tags, ngram_range=(1, 4))
\end{lstlisting}

\subsubsection{Relevance-Based Tag Selection}
After obtaining the TF-IDF scores, the system selects the top five most relevant tags for each narrative. These tags are considered the most descriptive and are used to summarize the key elements of each narrative.

\begin{lstlisting}[language=Python]
df['Top_5_Relevant_Tags'] = [find_top_relevant_tags(tfidf_matrix[i, :]) for i in range(tfidf_matrix.shape[0])]
\end{lstlisting}

\section{Results and Output}
The final output is an Excel file that includes:

\begin{itemize}
    \item The original dataset augmented with a new column called 'Top_5_Relevant_Tags.' This column contains up to five of the most relevant tags for each narrative.
    \item Additional sheets that list all the unique tags, as well as the frequent bi-grams, tri-grams, and four-grams identified in the narratives.
\end{itemize}

\section{Conclusion}
This project successfully automates the tagging process for a database of incident learning narratives. By employing advanced NLP techniques, the system is capable of summarizing, categorizing, and facilitating the analysis of these critical incident reports.

\end{document}
